\documentclass[12pt]{article}
\usepackage{amsmath}
\usepackage[top=1in, bottom=1in, left=1in, right=1in]{geometry}
\usepackage{multicol}
\usepackage{wrapfig}
\usepackage{listings}
\usepackage{enumerate}
\setlength{\columnsep}{0.1pc}

\title{CSE 3504: Project - Part 1}
\author{Joseph Sweeney and Soumya Kundu}
\date{\today}

\begin{document}

\maketitle
\vspace{-0.3in}

\paragraph{Practice 1:}
	\begin{enumerate}
	\item The path length is $76$
	\item With $100$ runs of the simulation:
		\begin{enumerate}
		\item The average path length is $69.23$
        \item The standard deviation of the path length is $45.02$
        \end{enumerate}
	\item With $100$ runs of the simulation using a $1000 X 1000$ lattice:
    	\begin{enumerate}
		\item The average path length is $74.08$
    	\item The standard deviation of the path length is $51.77$
        \end{enumerate}
	\end{enumerate}

\smallskip

\paragraph{Practice 2:}
	\begin{enumerate}
    \item The distance between the start point and the end point is $24$
	\item The lattice size is $208$
    \item With $10$ runs of the simulation:
    	\begin{enumerate}
    	\item Distance:
        	\begin{enumerate}
        	\item Mean is $14.6$
            \item Median is $6$
            \item Variance is $251.04$
            \item Standard Deviation is $15.84$
        	\end{enumerate}
       	\item Lattice Size:
        	\begin{enumerate}
        	\item Mean is $245.1$
            \item Median is $66$
            \item Variance is $140614.89$
            \item Standard Deviation is $374.99$
        	\end{enumerate}
    	\end{enumerate}

\pagebreak

 	\item With $100$ runs of the simulation:
    	\begin{enumerate}
    	\item Distance:
        	\begin{enumerate}
        	\item Mean is $15.63$
            \item Median is $13$
            \item Variance is $131.49$
            \item Standard Deviation is $11.47$
        	\end{enumerate}
       	\item Lattice Size:
        	\begin{enumerate}
        	\item Mean is $223.29$
            \item Median is $130$
            \item Variance is $73434.01$
            \item Standard Deviation is $270.99$
        	\end{enumerate}
    	\end{enumerate}
 	\item With $1000$ runs of the simulation:
    	\begin{enumerate}
    	\item Distance:
        	\begin{enumerate}
        	\item Mean is $15.20$
            \item Median is $12$
            \item Variance is $134.67$
            \item Standard Deviation is $11.60$
        	\end{enumerate}
       	\item Lattice Size:
        	\begin{enumerate}
        	\item Mean is $222.22$
            \item Median is $135$
            \item Variance is $66287.33$
            \item Standard Deviation is $257.46$
        	\end{enumerate}
    	\end{enumerate}
   	\item Between multiple runs of the script, with each run containing $10$, $100$, and $1000$ runs of the simulation, the mean, median, variance, and standard deviation of both the distance and the lattice size have less variation when the number of runs of the simulation is higher. Therefore, those numbers vary much more between different runs of the script for the case of $10$ runs of the simulation than they do for $1000$ runs of the simulation.

	\end{enumerate}

\pagebreak

\paragraph{Question 3:}

\paragraph{} Self-avoiding random walks pose interesting and often difficult problems in the fields of mathematics, probability theory, and enumerative combinatorics. However, their applications are also prevalent in numerous fields of scientific research. One such application is its use for protein structure prediction. The problem of protein structure prediction involves making inferences about the shape of a protein given its sequence of amino acids. Since the structure of a protein influences its function, accurately predicting protein structures can provide great insights into their functions, which can be extremely useful when investigating the role of such proteins in certain diseases. For low resolution backbone structure prediction of a protein, a two or three dimensional hydrophobic-hydrophilic square lattice model is used, where hydrophobic interactions are supposed to dominate protein folding. This happens as the protein core is formed by hydrophobic amino acids, while the hydrophilic amino acids tend to move to the outer surface due to their affinity with the solvent.

\paragraph{} Using this model, the protein structure prediction problem can be reduced to an optimization problem with a self-avoiding walk (SAW) on a two or three dimensional lattice, depending on the resolution. The goal is to find a SAW such that the free energy of the protein, which depends on the topological neighboring contacts between hydrophobic amino acids that are not contiguous in the primary structure, is minimal. For finding the structure of a protein with an amino acid sequence of length n, we need to find the most optimal n-step SAW. However, given all the n-step self-avoiding walks and the sequence of hydrophobicity of a protein, finding a minimal conformation of the protein is very difficult. In fact, it has been proven that this problem is NP-hard. As a result, instead of trying to develop exact algorithms to solve this problem, researchers have developed heuristics to predict the most probable conformation of the protein using self-avoiding random walks. These heuristics simulate protein folding in the real biological world by folding SAWs in order to minimize the free energy of the associated conformation.

\paragraph{} The application of self-avoiding random walks for protein structure prediction varies a lot from the simulations that were done in the previous two practices. First, the SAWs used for protein structure prediction must minimize a given cost function in order to accurately predict the true structure of the protein. In our simulations, no such optimization problem was considered. Next, the elements of the SAWs in the application discussed above are not homogeneous, as some of them are labeled as hydrophobic while others are labeled as hydrophilic, according to the given protein, to represent the two types of amino acids and their properties that are relevant to protein structure prediction. On the other hand, in our simulations, all elements of the SAWs were homogeneous. Finally, for the protein structure prediction problem, SAWs are also used on three dimensional lattices. However, our simulations were restricted to only two dimensional lattices.

\pagebreak

\paragraph{References}

\begin{enumerate}

\item Bahi, J.M., Guyeux, C., Nicod, J., \& Philippe, L (2013). Protein structure prediction software generate two different sets of conformations. Or the study of unfolded self-avoiding walks.

\item Bhansali, R., \& Tangerman, F. (2014). Are there Unfoldable Proteins in Dimension Three?

\end{enumerate}

\end{document}
